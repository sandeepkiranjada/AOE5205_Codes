
% This LaTeX was auto-generated from MATLAB code.
% To make changes, update the MATLAB code and republish this document.

\documentclass{article}
\usepackage{graphicx}
\usepackage{color}

\sloppy
\definecolor{lightgray}{gray}{0.5}
\setlength{\parindent}{0pt}

\begin{document}

    
    \begin{verbatim}
function f = ffunctaircraft03(t,x,m,S,CLalpha,CD0,oneoverpiARe,...
                              tinhist,Thist,alphahist,phihist)
%
%  Copyright (c) 2019 Mark L. Psiaki.  All rights reserved.
%
%  This function implements a nonlinear dynamic model
%  of a point-mass airplane flying over a flat Earth
%  in an atmosphere whose air density decays exponentially
%  with altitude.  This function models the effects of
%  time-varying thrust, angle-of-attack, and roll/bank-angle
%  inputs.
%
%  The dynamic model takes the form:
%
%    xdot = f(t,x)
%
%  where xdot is the time rate of change of the 6-by-1
%  state vector x and where the 6-by-1 vector function
%  f(t,x) is the output of this Matlab function.
%
%  Note:  The aerodynamic model does not include stall.
%
%  This particular function models the coordinate frame as
%  being centered at the center of the runway of the Blacksburg,
%  VA airport, which has coordinates latitude = 37.2076389 deg,
%  longitude = -80.4078333 deg, altitude = 649.7 m.  While it
%  uses a flat-Earth (i.e., constant) gravity field, its
%  constant gravity takes into account the Earth's J2
%  oblatness effect at the coordinate system center and
%  it subtracts off the centrifugal acceleration of the
%  coordinate system center as caused by the Earth's rotation.
%  This model also accounts for the additional centrifugal
%  effects of the Earth's rotation as caused by rotation of
%  the reference frame and any position offset from the
%  original of the reference frame.  In addition, it accounts
%  for the Coriolis acceleration that is caused by the Earth's
%  rotation.
%
%
%  Inputs:
%
%    t                     The time, in seconds, at which f is
%                          to be computed.
%
%    x                     = [X;Y;Z;V;gamma;psi], the 6-by-1 state
%                          vector of this system.  The first three
%                          elements give the Cartesian position
%                          vector of the aircraft's center of
%                          mass in local coordinates, in meters
%                          units, with X being the northward
%                          displacement from a reference position,
%                          Y being the eastward displacement from
%                          a reference position, and -Z being the
%                          altitude above the Blacksburg, VA
%                          airport (so that a positive value
%                          of x(3,1) indicates
%                          flight below the altitude of the
%                          Blacksburg airport.  The fourth element
%                          of x is airspeed (and the Earth-relative
%                          speed assuming no wind) in meters/second.
%                          The fifth element is the flight path
%                          angle in radians.  The sixth element is
%                          the heading angle in radians (0 is due
%                          north, +pi/2 radians is due east).
%
%    m                     The aircraft mass in kg.
%
%    S                     The wing area, in meters^2, which is
%                          the aerodynamic model's reference area.
%
%    CLalpha               The lift curve slope, dCL/dalpha, which
%                          is non-dimensional.
%
%    CD0                   The drag at zero lift, which is non-
%                          dimensional.
%
%    oneoverpiARe          = 1/(pi*AR*e), where AR is the non-
%                          dimensional aspect ratio of the wing
%                          and e is the Oswald efficiency factor.
%                          This composite input quantity is non-
%                          dimensional.  It is the coefficient
%                          of CL^2 in the drag coefficient model.
%
%    tinhist               = [tin0;tin1;tin2;...;tinM], the
%                          (M+1)-by-1 vector of times, in seconds,
%                          at which the airplane control inputs in
%                          Thist, alphahist, and phihist are
%                          defined.  This must be a monotonically
%                          increasing vector.  Also, it is required
%                          that tinhist(1,1) = tin0 <= t <= tinM = ...
%                          tinhist(M+1,1).  Otherwise, an error
%                          condition will occur.
%
%    Thist                 = [T0;T1;T2;...;TM], the (M+1)-by-1 vector
%                          of thrust inputs that apply at the times
%                          in tinhist, in Newtons.
%
%    alphahist             = [alpha0;alpha1;alpha2;...;alphaM], the
%                          (M+1)-by-1 vector of angle-of-attack
%                          inputs that apply at the times in tinhist,
%                          in radians.
%
%    phihist               = [phi0;phi1;phi2;...;phiM], the (M+1)-by-1
%                          vector of roll/bank-angle inputs that apply
%                          at the times in tinhist, in radians.
%
%                          Note: a piecewise cubic hermite
%                          interpolating polynomial is used
%                          to interpolate between times in tinhist
%                          in order to compute the thrust, angle-of-
%                          attack, and roll/bank angle that apply at
%                          time t.  These interpolations are computed
%                          using interp1.m.
%
%  Outputs:
%
%    f                     = [Xdot;Ydot;Zdot;Vdot;gammadot;psidot],
%                          the 6-by-1 vector that contains the
%                          computed time derivatives of the state
%                          from the kinematics and dynamics models
%                          of the aircraft.  f(1:3,1) is given
%                          in meters/second.  f(4,1) is given in
%                          meters/second^2, and f(5:6,1) is given
%                          in radians/second.
%

%
%  Compute the thrust, angle-of-attack, and roll/bank-angle
%  inputs that apply at time t.  It is more
%  efficient to do all three piecewise cubic hermite
%  interpolations simultaneously, as is done here.
%
   alphaTphi = interp1(tinhist,[alphahist,Thist,phihist],t,'pchip');
   alpha = alphaTphi(1,1);
   T = alphaTphi(1,2);
   phi = alphaTphi(1,3);
%
%  Compute the air density using a decaying exponential
%  model.  This model is good to about 1500 m altitude
%  (about 5000 ft).  This model recognizes that - x(3,1) + 649.7
%  is the aircraft altitude above sea level in meters.
%
   rho_sealevel = 1.225; % kg/m^3
   hscale = 10230.;      % meters
   haltitude = - x(3,1) + 649.7;
   rho = rho_sealevel*exp(-haltitude/hscale); % kg/m^3
%
%  Determine the airspeed.
%
   V = x(4,1);
%
%  Determine the dynamic pressure.
%
   qbar = 0.5*rho*(V^2);
%
%  Compute the lift and drag coefficients.
%
   CL = CLalpha*alpha;
   CD = CD0 + (CL^2)*oneoverpiARe;
%
%  Determine the lift and drag forces.
%
   qbar_S = qbar*S;
   L = qbar_S*CL;
   D = qbar_S*CD;
%
%  Set the flat-Earth gravitational acceleration at the
%  Blacksburg airport minus the effects of centrifugal
%  acceleration at the Blacksburg airport due to the
%  Earth's rotation vector, i.e., at the coordinate frame
%  origin.
%
   g = 9.79721; % meters/second^2
%
%  Compute the kinematics part of the model.
%
   cospsi = cos(x(6,1));
   sinpsi = sin(x(6,1));
   cosgamma = cos(x(5,1));
   singamma = sin(x(5,1));
   V_cosgamma = V*cosgamma;
   Xdot = V_cosgamma*cospsi;
   Ydot = V_cosgamma*sinpsi;
   Zdot = - V*singamma;
%
%  Compute the additional centrifugal acceleration term due
%  to the Earth's rotation rate vector and any offset of the
%  aircraft position from the coordinate frame center.
%  Also determine the Coriolis acceleration term due
%  to the Earth's rotation rate vector and any velocity
%  of the aircraft.  Compute these two accelerations in
%  North/East/Down coodinates.
%
   omegaEarth_ned = [5.80780002889625e-05;0;-4.40958072123465e-05];
   r_ned = x(1:3);
   v_ned = [Xdot;Ydot;Zdot];
   a_deltacentrifugal_ned = cross(omegaEarth_ned,cross(omegaEarth_ned,r_ned));
   a_coriolis_ned = 2.*cross(omegaEarth_ned,v_ned);
%
%  Transform the net acceleration of the delta centrifugal and
%  Coriolis terms from North/East/Down coordinates into
%  navigation axes.
%
%
%  Compute the gamma rotation matrix about the jhat axis.
%
   R2atgamma = [cosgamma 0 -singamma;0 1 0;singamma 0 cosgamma];
%
%  Compute the yaw rotation matrix about the khat axis.
%
   R3atpsi = [cospsi sinpsi 0;-sinpsi cospsi 0;0 0 1];
%
%  Complete the transformation from ned to nav axes.
%
   a_centrifugalcoriolis_ned = a_deltacentrifugal_ned + a_coriolis_ned;
   a_centrifugalcoriolis_nav = R2atgamma*R3atpsi*a_centrifugalcoriolis_ned;
%
%  Compute the dynamics part of the model.
%
   cosphi = cos(phi);
   sinphi = sin(phi);
   cosalpha = cos(alpha);
   sinalpha = sin(alpha);
   oneoverm = 1/m;
   Vdot = oneoverm*(T*cosalpha - D) - g*singamma - ...
            a_centrifugalcoriolis_nav(1);
   T_sinalpha_plus_L = T*sinalpha + L;
   gammadot = (1/V)*((cosphi*oneoverm)*T_sinalpha_plus_L - g*cosgamma + ...
                      a_centrifugalcoriolis_nav(3));
   psidot = (1/V_cosgamma)*((sinphi*oneoverm)*T_sinalpha_plus_L - ...
                            a_centrifugalcoriolis_nav(2));
%
%  Assemble the computed state time derivative elements
%  into the output vector.
%
   f = [Xdot;Ydot;Zdot;Vdot;gammadot;psidot];
\end{verbatim}

        \color{lightgray} \begin{verbatim}Not enough input arguments.

Error in ffunctaircraft03 (line 133)
   alphaTphi = interp1(tinhist,[alphahist,Thist,phihist],t,'pchip');
\end{verbatim} \color{black}
    


\end{document}
    
